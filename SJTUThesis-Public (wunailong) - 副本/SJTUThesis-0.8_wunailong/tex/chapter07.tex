%# -*- coding:utf-8 -*-
%!TEX root = ../thesis.tex

% \begin{summary}
\chapter{总结与展望}

\label{chap:conclusion}
\section{全文总结}

本文是针对水下机器人进行建模与非线性控制时所遇到的挑战,展开相关的研究工作,分别从理论和仿真实验的角度对不同类型的水下机器人的动力学建模、推力布置、数学模型的结构搜索与参数辨识、水流环境识别、控制应用导向的水下机器人建模以及考虑水下机器人非线性特点的鲁棒自适应控制展开分析,主要工作如下:

1. 欠驱动、全驱动型水下机器人的动力学建模及推力布置

(1) 依据机器人中的动力学系统驱动类型的定义,对完全处于水下的水下机器人进行系统判定;将水下机器人分为欠驱动性水下机器人和全驱动型水下机器人,并引入相关理论知识,为后续水下机器人的进行动力系统布置和控制方法选择提供了动力学模型理论。重点分析了作业型全驱动ROV和欠驱动型鱼雷形状AUV的动力学数学模型。

(2) 针对全驱动型水下机器人的动力布置优化问题,基于推进器的空间布置,给出了不同形式的推力分配模型。分析建立的基于运动学的推力分配模型,提出面向控制的推力控制向量,着重利用对运动影响性强的推进器,并使用水下机器人的PID姿态控制来进行实例化分析。

2. 水下机器人的模型搜索、辨识及典型水流环境的识别

(1) 针对水下机器人尤其是欠驱动型水下机器人的模型探索与辨识问题,重新审视水下机器人的模型的辨识过程,将水下机器人的描述分为模型的数学结构与模型参数两个方面,重新提出可以进行水下机器人模型结构与参数共同搜索的符号回归方法。

(2) 使用信噪比(SNR)模型获取受到不同噪声影响的数据集,分别使用Levenberg-Marquardt方法、遗传算法、基于遗传编程技术的符号回归算法对模型参数进行估计,经过与阻尼系数的真实值对比,验证了符号回归方法在模型探索和参数辨识应用的有效性,并可以用于发现蕴藏于数据集中的新现象。

(3) 为辨识水下机器人的工作典型水流环境,从利于水下机器人认识水下环境的角度,基于鱼类体线感知水流的生物机理,采用机器学习方法研究水下机器人周布压力传感器阵列识别水流模式的热点问题。提出一种基于线性判别分析和支持向量机相结合的分类器来辨识水流方向。使用流体力学数值分析方法获取反映水下机器人周围模型水流变化的压力数据,并使用线性判别分析方法压缩压力感知以获取最优的特征向量矩阵,使用最优的SVM内核函数对水流的流向进行了预测,并使用拟合方法估计流速。

3. 控制导向的水下机器人建模

(1) 针对水下机器人控制应用对水下机器人建模的要求,分别从水下机器人外形、数学模型两个角度对水下机器人进行建模, 求解水下机器人模型中的惯性质量矩阵、附加质量矩阵和阻尼等关键项,给出多种不同的方法用于水下机器人的建模,最大程度方便了控制应用。

(2) 经验法:考虑到水下机器人的具体设计模型未知且实验数据不可获取,仅仅具有水下机器人的一些设置功能要求以及物理模型的基本参数,尤其对于非初等几何表面的低速水下机器人,通过分析水下机器人动力学模型的各个项的重要性,基于实验数据的经验总结,可以对水动力参数进行相对精确的快速估算。

(3) 数值分析法:考虑到水下机器人的CAD模型已知,但是实验数据未知,处于设备开发前期,对于形状复杂的水下机器人,主要使用流体计算软件STAR CCM+和ANSYS/AWQA软件分别求取模型中的阻尼与附加质量关键参数项,可以快算且精确地计算水下机器人中的模型关键项。

(4) 针对水下机器人非线性精确模型已知,对运动方程进行解耦,并进行泰勒展开,确定可以用于自适应控制的参考模型,给出了俯仰自由度的泰勒展开模型。

4. 水下机器人的鲁棒自适应控制及非线性瞬态优化

(1) 自适应控制的必要性:针对具有高度非线性和多自由度耦合特点的欠驱动型水下机器人的控制问题,分别对水下机器人中系统非线性和模型不确定性进行分析,确定出模型结构不确定、参数不确定性、未建模动态为水下机器人控制主要影响因素,使用非线性自适应控制方法作为本文的控制理论基础。

(2) 鲁棒控制的必要性:系统地给出水下机器人控制系统稳定的相关理论,基于水下机器人模型自适应控制的必要性,给出闭环系统的状态模型参考自适应控制。考虑实际系统中经常存在的有界干扰问题,分析非线性自适应控制的不稳定因素,使用射影算子理论对状态模型参考自适应控制方法中的自适应增益进行修正,确保自适应控制在存在有界干扰时不出现参数漂移问题。

(3) 针对模型参考自适应控制中的自适应更新慢计算耗时长、参数漂移问题,引入能够将鲁棒性、自适应和收敛速度解耦的$L_1$自适应控制方法。考虑噪声干扰、时间延迟以及瞬间扰动的意外情况下,将$L_1$自适应控制器用于REMUS AUV的6自由度非线性模型的俯仰自由度控制,分析水下机器人跟踪随意设定的期望指令的响应结果,确定L1自适应控制的可行性。

(4) 针对欠驱动水下机器人REMUS AUV的驱动器具有动态特性时的鲁棒自适应控制问题,考虑到水下机器人模型的不确定性和扰动,鲁棒MRAC和$L_1$自适应框架被选择为主要的控制框架;介绍水下机器人实际系统中驱动器的输入阈值、死区、时间延迟的驱动器非线性特点,确定欠驱动型REMUS AUV中舵片的输入阈值为主要影响因素。为应对驱动器的输入阈值的挑战,选择使用Riccati补偿器来优化MRAC和$L_1$自适应控制中的瞬态响应问题,并将带有驱动补偿器的鲁棒自适应控制方法分别用于时变线性系统和具有噪声扰动的REMUS AUV的6自由度耦合模型中,姿态控制实验和深度轨迹追踪实验的结果验证了所提出的带有驱动补偿器的鲁棒自适应控制的可行性,并从欠驱动机器人的研究角度进一步利用了系统动态特性改善$L_1$自适应控制。


\section{论文创新点}

(1) 提出一种启发式模型探索方法。通过对水下机器人数学模型的参数辨识的实现前提以及辨识过程进行分析,结合基于基因编程的符号回归技术,提出一种新的水下机器人模型确定方法;所提出的符号回归方法不仅用于辨识水下机器人的模型参数,还可以打破已有的认识偏见,自动的探索水下机器人的模型结构,发现新现象,给出模型的另一种数学表达。

(2) 提出一种水流环境感知方法。通过对水下鱼类生物进行生理性分析,发现侧线器官对于鱼类感知水下环境起着至关重要的作用。分析水下机器人的主要水下环境,根据建立的流体动力学模型,基于鱼类的侧线感知水流机理,利用压力传感器网络的数据对水下机器人环境中的水流进行识别,有效地提高水下未知环境的认知。

(3) 提出一种面向控制应用的水下机器人建模方法。针对水下机器人的控制应用中模型难以确定的问题,基于水下机器人的三维机械模型以及结合建立水下机器人运动学与动力学模型,分析出对于水下机器人控制重要的关键项,测算出主要特征参数,分别使用经验法与流体数值分析法,使用MATLAB、STAR CCM+、AQWA软件求水下机器人的附加质量项、阻尼项,通过对比发现,所提出的方法可以很好的用于水下机器人模型尤其是复杂形状模型的控制应用导向的水下机器人建模;使用CFD计算软件,基于控制应用,建立精准的水力学模型。

(4) 提出一种可以应对水下机器人多种系统动态的鲁棒快速自适应方法。考虑影响水下机器人运行性能的模型参数不确定性、模型结构不确定性、非线性项、有界干扰以及驱动器的非线性特性,分析传统自适应控制中的不稳定问题,使用射影算子对于非线性自适应控制增益进行优化,确保控制器的有界干扰的稳定性;对水下机器人驱动器的非线性进行具体描述,分析其带来的耦合扰动与稳态误差问题,使用基于Riccati方程的现代补偿器对鲁棒自适应控制方法进行优化,提高水下机器人应对瞬态扰动的能力,也挖掘了驱动的潜力。



\section{展望}
% \end{summary}
论文是关于水下机器人建模与非线性自适应控制的研究与分析,从理论和仿真分析的角度解决了一些不同类型水下机器人的动力学建模、模型探索与辨识、控制导向应用的水下机器人建模以及带有有界干扰和驱动器阈值等非线性情况的欠驱动水下机器人的鲁棒自适应控制问题,得到了一些对于水下机器人研究者有一定参考价值的方法;但从水下机器人所及的建模与控制的方方面面而言,本文所做的研究工作仍然有限,还有很多有值得深入探究的问题要解决,并基于此展开水下机器人的建模与控制有关工作:

(1) 水下机器人的推进器布置对于水下机器人的动力布置建模自适应与控制自适应的问题。推进器的分配布置多是采用数学的形式预先加载,文中虽然尝试性提出了推力矢量来描述每个人推进器的影响性,并进行先验的设定出每个推进器的各个自由度的影响系数;这种加载推进器的方法虽然比以往推力分配矩阵整体加载有所改进,但是仍然离智能机器人的需要有很大的差距;所期待智能机器人是有着自动识别每个推进器对本体影响性的能力,可以通过智能学习的方法预测出每个推进器的各个自由度的系数并根据运动控制目标对所提出的系数进行再优化;智能加载动力单元可以加快水下机器人的设计与开发,这块的研究与深度学习以及三维建模虚拟技术相结合应该能提升水下机器人的智能水平。

(2) 水下机器人对于水下环境的感知与机器人水下定位的数据融合问题。水下机器人运行的水下环境多是被视为外界干扰来处理,文中探讨性地研究了水下机器人感知水流环境的问题,并利用机器学习方法使用流体数值分析获取的水下机器人的表层压力数据训练分类识别器;这种基于鱼类侧线感知水下环境的能力虽然让水下机器人真正感知水下世界有了理论基础,但是在实际的感知系统搭建上仍有许多的挑战;水下机器人感知所需的数据越是密集,那么它对水下世界的识别越是完善,但是这种高密度分布式数据网络对于数据的采集、传输、预处理以及识别的硬件与通信网络都带来了挑战。以往水下机器人的由于将水流环境视为干扰,在进行水下定位时,是没有融合环境的信息,文中尝试性的提出了一个融合框架,如果结合传统声学与惯导水下定位技术应该能为数学融合提供更多的校正信息,提高水下机器人的水下感知能力。

(3) 水下机器人控制导向的流体建模求取附加质量项与智能几何结构拓扑相互结合的问题。为对水下机器人进行建模求取出关键项用于控制,采用的估算法与流体数值分析法虽然可以一定程度上的预测模型参数,但是在附加质量项的参数计算上仍存在一定问题,该问题是复杂装配体模型的简化问题。采用人工的简化虽然可以实现目标,但并不一定是最优的,且该方法耗费时间;智能几何结构拓扑技术在三维结构演化上有着很大的潜力,首先将模型的结构分解为各个闭集几何体,对于被包含的几何体可进行合并,只保留较大的集合;对于较小的集合体,如螺丝钉、螺母,可以进行过滤处理;预处理后可以进行结构的拓扑获得最佳的结构性模型简化;最后可以采用面集合平滑技术对结构拓扑后的模型进行处理,将智能结构拓扑结束与模型分析相结合,应可以提高分析的效率与质量。

(4) 在存在网络时间滞后、通信丢包等非线性特性时的水下机器人的非线性控制问题。本论文的研究探索了一般不确定性如有界干扰以及驱动器尤其是舵片的输入阈值饱和存在时的控制问题,虽然在鲁棒自适应控制可以应对有界干扰的一般不确定,但是自适应控制并非所有不确定性的万能解药;实际中水下机器人系统与理论仿真中水下机器人系统是存在很大的模型误差的,而网络时间滞后、通信丢包存在的情况下,水下机器人的控制系统需要能够应对以上因素带来的动态扰动,这需要设计出优秀的控制器来克服上述非线性问题。

(5) 智能水下机器人的自适应控制源域在不同的本体系统进行迁移的问题。本论文研究的控制方法是针对单体水下机器人而设计的自适应控制器,但是这种基于参考模型的控制器在某种程度上限制了控制器从一个水下机器人迁移到另一个水下机器人;一个好的自适应控制器不仅仅能针对单体进行处理,还应该可以从一个单体的控制设计时建立一个自适应源域,利用迁移学习技术将一台水下机器人具有的能力迁移到另一台水下机器人上,实现水下机器人的智能开发与学习,进行扩大自适应控制的应用范围;同时也能大幅度降低开发成本,使得水下机器人具有学习能力。
