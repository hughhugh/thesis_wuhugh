%# -*- coding:utf-8 -*-
%!TEX root = ../thesis.tex
%%==================================================
%% abstract.tex for SJTU Master Thesis
%%==================================================

\begin{abstract}
\addcontentsline{toc}{chapter}{摘要}
无人水下机器人是人类探索海洋深处的一种有效工具,其在海洋中进行结构物或设施监测与作业的活动对水下机器人系统运行提出更加准确与稳定的要求。认识和操控水下机器人,尤其是欠驱动型水下机器人并不容易,无论是从本质上还是从便于实际操控上。水下机器人模型用来解释运动和施加在机器人上的力的关系,目前还没有一种通用的方法可以自动探索水下机器人本体模型的结构和参数。而由于机器人自身的驱动布置、形态重量和环境要素,水下机器人系统在水下环境中或者进行动态运动时呈现时变演化特点。并且,水下机器人所处的环境变化快、载体需可携带传感器设备、驱动器具有非线性以及一些意想不到的扰动与障碍都会使得控制系统运行变得不稳定。 因此, 探索性地研究水下机器人的模型与环境,并设计能够适应系统中的非线性、不确定性的鲁棒自适应方法就对水下机器人的理论和实践方面非常有必要。

本文围绕水下机器人模型的启发式搜索与非线性问题的鲁棒自适应控制进行展开,研究的重点放在具有复杂表达形式的水下机器人数学方程的结构与参数辨识以及考虑实践中的水下机器人的模型不确定性与驱动器非线性特点的水下机器人快速鲁棒自适应控制上。通过本文的研究,提出了一些有趣的水下机器人系统模型探索与系统控制方法,并进行大量的理论仿真对比与实验验证,为水下机器人的应用提供了理论与技术基础。本文的主要建模与控制贡献如下:

(1)针对水下机器人非线性系统的6自由度模型的结构与参数构成,提出了基于符号回归模型构建与探索方法。介绍了运动模型方程的基因树表达方法,并给出了符号回归中的模型表达树的演化过程。摒除了人类在研究模型的偏见与领域知识限制,揭示了蕴含在水下机器人运动数据集中的内在关系,让数据本身解释模型。从模型参数和结构两个角度,依赖获取的实验数据,建立精确的数学方程表达,发现新的模型结构,为认识与解决机器人的控制问题提供理论基础。

(2)考虑到水流环境的重要性,针对鱼雷型水下机器人不同的水流环境,基于鱼类侧线感知水流的机理,使用流体动
力学模拟载体受到的水流工况,选择使用线性判别分析的压缩感知处理方法进行数据降维、使用支持向量机分类方法训练并建立水流方向感知分类模型,并提出水流估计与惯导的融合框架,从功能角度验证了仿生侧线感应水流的能力,为水下机器人主动利用海洋环境提供理论基础。

(3)针对开发高精度模型的困难,并考虑到鲁棒控制目标应用中模型不确定项的影响,分别对两种不同的水下机器人(ROV、AUV)进行建模,其中ROV模型具有非初等几何表面形状复杂。使用计算流体软件分别计算流体动力学模型中的关键参数:附加质量项和阻尼矩阵项。针对具有理论公式描述的水下机器人系统,目标为自适应控制应用,使用泰勒展开方式对运动方程进行简化。给出了从外形和从数学两种不同的控制应用参考模型的确定方法。

(4)考虑到水下机器人系统特性:模型不确定性、非线性、各个自由度的耦合性、环境(盐度,机械冲击)中出现的干扰,给出了鲁棒自适应控制方法的基本控制架构以及稳定性判据。针对具有静不稳定特点的非线性6自由度水下机器人系统的俯仰控制问题,分别使用基于射影算子的模型参考自适应控制(MRAC)方法与 $L_1$ 自适应控制方法设计控制器。 通过控制实验对比,发现$L_1$ 自适应控制在响应速度和控制稳定效果上的优异性。


(5) 针对实际水下自治水下机器人平台控制中的时变模型参数以及驱动器的输入阈值、死区以及时变延迟的非线性特点进行理论分析,并基于模型参考自适应控制和$L_1$ 自适应控制进行设计可优化控制输出的抗饱和补偿器,将其用于水下机器人控制。实现了对静不稳定6自由度的自治水下机器人的俯仰、深度自由度的控制,验证了使用抗饱和补偿器优化鲁棒自适应控制方法的可行性。考虑多模式噪声与瞬态干扰、并与以往实验中使用的控制器进行对比,验证带有驱动补偿的$L_1$ 自适应控制方法的稳健而快速的控制性能。

本文主要展示了在水下机器人的建模与非线性自适应控制方面的理论研究,可以用于多种航行器系统模型的研究中,也可为人类在海洋活动中研究并应用水下运载器提供一种新的视角。

\keywords{\large 水下机器人 \quad 符号回归 \quad 系统辨识 \quad 模型探索 \quad 控制导向建模 \quad
$L_1$自适应控制 \quad MRAC \quad 抗饱和补偿器 }
\end{abstract}
