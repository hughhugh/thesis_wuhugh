%# -*- coding:utf-8 -*-
%!TEX root = ../thesis.tex
\begin{thanks}

到了这个时候,言少感多。感谢培养了我五年零两个月的上海交通大学,学校的校园宽大而美丽,承载了交大的一草一木,也承载了浓厚学术科研氛围中的有志青年,让我感受到了努力者的世界。失败与成功,沮丧与喜悦,无数次的跌宕让我更加坚强,也让我用不同的眼睛看世界。博士论文的写作期间是在江南的一个雨季。沥雨淼淼,伴随着心中的激腾,不知不觉中看到了多个黎明。焦虑彷徨,不断地晚夜书写,才可以让心中点点写在了今天。

感谢我的导师葛彤教授,葛老师严谨的学术作风、执着的科研精神与精益求精的人格品质,将会使我一生获益。博士期间,在葛老师的指导与支持下,我对理论探索与工程实践的结合研究都有了新的认识,并将此后作为进行科学研究的指导准则。

感谢吴超老师、王旭阳、赵敏老师对我进行科学研究的支持。每一次研究上的推进都有你们背后的关心和投入。感谢杨德庆教授对我学习研究进度的关心,研究上的讨论都极大地启发我进行科研探索。

感谢船舶与海洋工程系水下工程所的全体老师与团队成员的支持与鼓励。感谢博士期间一起奋斗的李翔、王涛、霍星星、刘宗霖、王健、刘义、于立伟、陈垣毅、宋磊建、欧阳义平、魏成柱同学,你们的努力的身影写进了我的年华。

感谢冯志光博士、杨睿博士对于我科研中遇到问题总是很无私解答与帮助。感谢ResearchGate,Github提供了一个好的交流平台,感谢Vicent、Pouria以及Jaxxzer(Jacob Walser),让我感受到你们思想的开放与对论文Beautiful的追求。

特别感谢我的父母与家人,对我的设身处地的理解与支持,你们是我最坚实的依靠,愿你们松柏长青、气愉神怿。感谢特别的你-计苓,风风雨雨,相依相伴。

希望每个人都保持身体健康。“如果生死存活仅仅靠心态来决定的话,那我肯定能再活五十年。然而不幸的是,我们的身体也有发言权。”

最后,向所有关心、支持、鼓励我的每一个人再次感谢。


\end{thanks}
