\begin{englishabstract}
\addcontentsline{toc}{chapter}{Abstract}
Unmanned Underwater Vehicle(UUV) is an effective tool for the human to explore and exploit the deep ocean, which proposes more accurate and stable requirements for the development of the vehicle system when vehicles are utilized in many activities of the human being to discover ocean like inspecting underwater infrastructures or deploying submarine equipment. It is not easy to understand the underwater vehicle, especially the under-actuated underwater robots, in nature or manipulate it precisely in the exploration underwater. Models for the underwater vehicle are utilized to explain the relationship between movements and forces exerting on the robot. Yet currently there are no general methods can arrive a better model with structure and parameters for vehicle automatically in the discovery of model datasets. And considering the thruster configuration, the designed shape, and weight as well as the unknown underwater environments, the system described in mathematics or dynamic shows time-varying evolution in the model. Moreover, changes in the underwater environment, sensor instruments or actuators carried as well as some unexpected disturbances and obstacles in the operation will make the control system become unstable. Therefore, it is necessary to discover the modeling of a vehicle, perceive underwater surroundings, and develop a suitable controller that can adapt to nonlinearities, uncertainties in the vehicle system from the perspective of theoretical research and control applications.

In this thesis, the heuristic search of the underwater vehicle model and the robust adaptive control scheme of vehicle system with highly nonlinearities are carried out.  The research focuses on the model structure discovery and the parameters identification of the underwater vehicle, such as Remotely Operated Vehicle(ROV) and Autonomous Underwater Vehicle(AUV), as well as on developing excellent robust adaptive control approach where the uncertainties of the vehicle model and limitations of actuators are both taken into account. In this paper, some interesting approaches that can model the underwater vehicle via discovering datasets and controllers are proposed. Results from numerical simulations and experiments are both demonstrated and discussed, which proves to be an appropriate solution in theory and technique for further application in water. Contributions to the modeling and control of a vehicle in this thesis are listed as the following:


(1) Aiming at the difficulties of determining the structure and parameters of a nonlinear 6 Degree-Of-Freedom(DOF) underwater vehicle system, the symbolic regression used to investigate the modeling of the vehicle is put forward. Gene trees that can describe the movement mathematical equation are introduced. The evolution process of the model expressed in gene trees using symbolic regression method is also given out. Discarding the human bias in the study of the modeling for a vehicle and the limitations of knowledge in the field of expertise, the intrinsic relationship hidden in the measured datasets of a vehicle is revealed intelligently, which empowers the data the ability of self-discovery. Exact mathematical equation with new model structure is obtained using experimental data via investigating the structure and parameters of the model, which provides a new approach to re-understand the vehicle system and the basis for future controller design.

(2) Considering the effects of the water flow situation on an underwater vehicle, especially UUV, the motion of the vehicle in water is simulated using computed fluid dynamic(CFD) technique based on the mechanism of the fish lateral line that can perceive water flow. Linear discriminant analysis(LDA) method in the view of the concept of compressed sensing is employed to reduce the dimensionality of data. Support vector machine approach is utilized to train the water flow direction classifiers and the framework that fusing water flow forecasting and an inertial navigation system is proposed. In the thesis, the ability of vehicle identifying current like the fish lateral line is investigated in theory, which also provides a new point of view for an underwater vehicle to exploit ocean.

(3) Aiming at the difficulty of developing more accurate vehicle model and taking into account the effects of model uncertainties in robust adaptive control, the modeling for a vehicle like ROV, the shape of which is complex and composed of non-elementary geometry surfaces, and AUV in torpedo shape are investigated, respectively. Key terms like the added mass and damping matrix in the hydrodynamic model of underwater vehicle are both determined using CFD software. Oriented for the adaptive control, the nonlinear formula of the vehicle is simplified by using Taylor expansion method. The methods of determining the reference model for control from the outside shape and the theory mathematical expression are both put forward.

(4) Considering the characteristics of an underwater vehicle system, which mainly includes the uncertainties of the model, nonlinearities, the coupling relationship of each DOF and the interferences from the environment(salinity, mechanical shock), the basic diagram of the robust adaptive control scheme as well as the stability criterion are introduced. Aiming at the pitch channel control of a 6 DOF underwater vehicle system with static instability in water, the projection based model reference adaptive control(MRAC), as well as $L_1$ adaptive control scheme, are both used to deal with this problem. Performance results of control are demonstrated and it is found that the $L_1$ adaptive control surpass MRAC whether in the response speed or in the adaptability of control approach to stabilize the vehicle system.

(5) With the purpose of coping with the time-varying model parameters of AUV as well as the nonlinearities of actuators like input saturations, dead-zone, and time-varying delay, the anti-windup compensator is extended into the structure of MRAC and $L_1$ adaptive control to optimize the controller outputs. The attitude and position control of a 6 DOF underwater vehicle in the pitch channel and the depth mode are studied and simulated, which proves the effectiveness of the proposed control scheme. The robust and adaptive performances of $L_1$ adaptive control with anti-windup compensator are demonstrated and compared with the MRAC controller utilized in the experiments, where measurements noise, unexpected disturbance and limitations of actuators are considered.

The study on modeling and nonlinear adaptive control of an underwater vehicle are presented in this work. The methods proposed can be used in the research of various systems, and can provide a new perspective for the human being to investigate an underwater vehicle in theory and apply vehicle in the future exploration.



\englishkeywords{\large Underwater vehicle \quad Symbolic regression \quad System identification \quad Model discovery \quad Control oriented modeling \quad $L_1$ adaptive control \quad MRAC \quad Anti-windup compensator}
\end{englishabstract}
