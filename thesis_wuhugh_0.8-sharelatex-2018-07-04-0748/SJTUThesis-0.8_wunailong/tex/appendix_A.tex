%# -*- coding:utf-8 -*-
%!TEX root = ../thesis.tex
%%==================================================
%% chapter02.tex for SJTU Master Thesis
%% Encoding: UTF-8
%%==================================================

\chapter{有关数学基础 }

% \section{定理}

% \subsection{}

\begin{thm}[惯性矩阵的科氏力-离心力矩阵参数化定理]
\label{app_A:thm:1}
定义 $6 \times 6$ 的惯性质量矩阵,且满足 $\bm{M} > 0$ ,惯性附加质量如下:

\begin{equation}
\label{eq:app:1}
\bm{M} = \begin{bmatrix}
		 \bm{M}_{11}   &  \bm{M}_{12}   \\
		 \bm{M}_{21}   &  \bm{M}_{22}   \\
		 \end{bmatrix}
\end{equation}

这样,可以获得满足$\bm{C}({\bm{\nu}}) = - \bm{C}({\bm{\nu}}) ^T$参数化的科氏力-离心力矩阵如下:

\begin{equation}
\label{eq:app:2}
\bm{C}({\bm{\nu}}) =   \begin{bmatrix}
                \bm{0}_{3 \times 3}      &  -\bm{S}( \bm{M}_{11} \bm{\nu}_{1} + \bm{M}_{12} \bm{\nu}_{2})      \\
    -\bm{S}(\bm{M}_{11}\bm{\nu}_{1}+\bm{M}_{12}\bm{\nu}_{2})   &  -\bm{S}(\bm{M}_{21}\bm{\nu}_{1}+\bm{M}_{22}\bm{\nu}_{2}) \\
					   \end{bmatrix}
\end{equation}
\begin{proof}

动能$\bm{T}$可以写成二次形式:
\begin{equation}
\label{eq:app:3}
{T} = \frac{1}{2} \bm{\nu}^T\bm{M}\bm{\nu}
\end{equation}

扩展此表达式可得:

\begin{equation}
\label{eq:app:4}
T = \frac{1}{2} ({\bm{\nu}_1 }^T \bm{M}_{11} \bm{\nu}_{1} + {\bm{\nu}_1 }^T \bm{M}_{12} \bm{\nu}_2 + {\bm{\nu}_2}^T \bm{M}_{21} \bm{\nu}_{1} + {\bm{\nu}_2}^T \bm{M}_{22} \bm{\nu}_2)
\end{equation}

所以,可以得到
\begin{eqnarray}
\label{eq:app:5}
\frac{\partial T}{\partial \bm{\nu}_1 } &=& \bm{M}_{11} \bm{\nu}_1 + \bm{M}_{12} \bm{\nu}_{2} \\
\label{eq:app:6}
\frac{\partial T}{ \partial \bm{\nu}_2} &=&  \bm{M}_{21} \bm{\nu}_{1} + \bm{M}_{22} \bm{\nu}_{2}
\end{eqnarray}

使用Kirchhoff方程\cite{fossen1994guidance},可以得到:
\begin{equation}
\label{eq:app:7}
\begin{aligned}
\bm{C}(\bm{\nu}) \bm{\nu} {\buildrel \Delta \over =}& \begin{bmatrix}
                                      {\bm{\nu}_2} \times {\frac{\partial T}{ \partial {\bm{\nu}_1}}} \\
{\bm{\nu}_2} \times \frac{ \partial T} { \partial {\bm{\nu}_2}} + {\bm{\nu}_1 } \times {\frac{\partial T}{\partial \bm{\nu}_1}}\\
\end{bmatrix} \\
=&\begin{bmatrix}
   \bm{0}_{3 \times 3} & -\bm{S} \frac{\partial T }{ \partial {\bm{\nu}_1}}   \\
   -\bm{S} \frac{\partial T }{ \partial {\bm{\nu}_1}}  & -\bm{S} \frac{\partial T }{ \partial {\bm{\nu}_2}}   \\
 \end{bmatrix}
 \begin{bmatrix}
    \bm{\nu}_1 \\
    \bm{\nu}_2 \\
 \end{bmatrix}
\end{aligned}
\end{equation}

将式\ref{eq:app:5}和式\ref{eq:app:6}代入式\ref{eq:app:7}即可得证。
\end{proof}
\end{thm}


%----------------------------------------------------------------------


\begin{thm}[补偿器稳定性与鲁棒性定理]
\label{app_A:thm:2}
一个全阶抗饱和补偿器
\begin{equation}
\Theta =
\begin{bmatrix}
\Theta _{1}^{'} \\
\Theta _{2}^{'} \\
\end{bmatrix}
\in R^{(m+q)\times m}
\end{equation}
如果存在矩阵 $P=P^{'}>0$,$W=diag(\omega_1,\ldots,\omega_m)>0$ 和 一个正实数 $\gamma$ 使得下面方程\ref{eq:app:riccati}和\ref{eq:app:Z}成立:
\begin{equation}
\label{eq:app:riccati}
\tilde{A}^{'}P+P\tilde{A}+PBR^{-1}B^{'}P+\tilde{Q}=0
\end{equation}
式中,
\begin{eqnarray}
\tilde{A}=A+BR^{-1}D^{'}C\\
\tilde{Q}=C^{'}(I+DR^{-1}D^{'})C\\
R=(\gamma^2 I - D^{'} D) > 0\\
\end{eqnarray}

\begin{equation}
\label{eq:app:Z}
Z=(2W  -  D^{'} D - \gamma^{-2} W^{2}) > 0
\end{equation}

如果方程\ref{eq:app:riccati} 和 不等式\ref{eq:app:Z}都被满足,就可以通过计算式\ref{eq:chap6:awks}中的矩阵增益 $F$(见式\ref{eq:chap6:F})的值来获得合适 $\Theta$ 使得 $\left \|  T_p   \right \|_{i,2} < \gamma$。

\begin{equation}
\label{eq:app:F}
F= -\gamma^2 (W^{-1} - \gamma ^{-2} I)R^{-1} (B^{'}P + D^{'} C)
\end{equation}

鲁棒性边界可以给出如下:
\begin{equation}
\label{eq:app:robustmargin}
\frac{1}{\mu} = \frac{1}{\gamma} \sqrt{\omega_p}
\end{equation}


\end{thm}
